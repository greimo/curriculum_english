\documentclass[12pt]{article}
\usepackage[latin1]{inputenc}
\usepackage[margin=2cm]{geometry}
\usepackage{amsmath}
\setlength{\parskip}{0.25cm}
\setlength{\parindent}{0mm}
\usepackage{graphicx}
\usepackage{graphics}
\newcommand{\HRule}{\rule{\linewidth}{0.5mm}}
\usepackage{enumerate}




\begin{document}

\title{Curriculum Vitae}
\author{Juan Jos� de Jes�s Cervantes Avila}
\date{October 2018}
\maketitle


\section{Personal Information}

	\begin{tabular}{ l l }
		\textit{Address} & Salaverry 854-206 Col. Lindavista CP. 07300 \\
		\textit{E-mail} & greimo1415@gmail.com  \\
		\textit{Cell phone number} & 5528791155 \\
		\textit{Date of birth} & March 21, 1987
	\end{tabular}


\section{Education}
	
	\begin{tabular}{ l l }
		2005-2010 & Bachelor of Science in Applied Mathematics, 90 \\
				  &	Instituto Tecnol�gico Aut�nomo de M�xico \\
				  &	Special Mention. \\
				  & \\
		2019 -    & Master of Science in Computer Science \\
				  & Georgia Institute of Technology \\
				  & Starting 2019 - Expected 2021 (Online)
	\end{tabular}


%\section{Scholarships and Awards}

%	\begin{enumerate}
%	\item ITAM Scholarship from 2005 to 2008
%	\item TOEFL Certificate
%	\item Final college average: 9
%	\end{enumerate}



\section{Work Experience}


\subsection{March 2014 - Present \hspace{0.5 cm} BBVA Bancomer}

\begin{itemize}
	\item Construction of a credit scoring model for admission of new credits for the Small and Medium Enterprises (SMEs) portfolio. The model consists mainly of logistic regressions and genetic algorithms for variables selection.
	\item Fitting of a predictive model of expected loss for the mortgage business unit. The model used was a multinomial logistic regression.
	\item Construction of a model for the computation of credit lines for SMEs credits. The model consisted of contraction and expansion factors based on empirical default rates.
	\item Development of a model to optimize the recovery process within the traditional credit cards portfolio. The model used linear and dynamic programming.
	\item Calculation of the LGD parameter for the traditional credit cards portfolio.
	\item Computation of classical credit risk parameters (PD, LGD, CCF) under the IRB model, the IAS regulation and more recently IFRS 9. All of these, for the SMEs portfolio.
\end{itemize}



\subsection{August 2013 - December 2013  \hspace{0.5 cm} Sm4rt Predictive Systems}

\begin{itemize}
	\item Partial development of a mixed effects credit scoring model for a mortgage portfolio.
	\item Writing of technical literature of a new model for credit card fraud detection. The model was based on logistic regressions and random forests.
	\item Partial development of a basic but big scale algorithm to find Points of Compromise of credit cards. Given the amount of data, the project was developed using Amazon Elastic Map Reduce tools such as Pig Latin.
\end{itemize}




\subsection{July 2010 - July 2013 \hspace{0.5 cm} Las Quince Letras}

\begin{itemize}
	\item Development of sales predictive models: mainly linear regressions using splines, mixed effects models and bayesian techniques.
	\item Market segmentation projects.
	\item Construction of models for classification problems (trees, random forests, support vector machines, logistic regressions and boosting).
	\item Analysis of causality using linear models and matching.
\end{itemize}




\section{Virtues}

I think one of my main virtues at work is succesfully combining the understanding of the mathematical / statistical models I work with, with the hability of presenting them in an gentle and accessible way to non-technical audiences. I also think I have very good skills for independent learning.

I consider myself a very responsible person and I am always committed to do my best effort and to work in an ethical and professional way.

	


\section{Computing}

\begin{itemize}
	\item Advanced data analysis with R and SAS.
	\item Intermediate knowledge of SQL, Matlab and Java. 
	\item Basic concepts of Python, C and Pig Latin.
	\item I am currently taking a course in data science fundamentals at BBVA Bancomer based on PySpark and Scala.
\end{itemize}




\section{Languages}
Spanish (Native), English (Advanced) and French (Basic)




\section{Academic Interests}

My interests span across many areas including mathematics, computer science and finance. Some particular areas of interest are machine learning and AI, theoretical computer science (algorithms, complexity and operations research), continous optimization and computational finance.


 


\end{document}